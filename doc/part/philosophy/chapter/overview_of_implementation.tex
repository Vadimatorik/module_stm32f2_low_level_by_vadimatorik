\chapter{Краткий обзор реализации библиотеки}
\begin{enumerate}
	\item Библиотека написана на C++14. 
	\item Большую часть библиотеки составляют constexpr функции, которые обрабатывают заполненные пользователем структуры инициализации периферии на этапе компиляции и создают маски регистров для всевозможных, указанных в структуре инициализации, режимов.
	
	В реальном времени созданные из const constexpr структур инициализации глобальные const constexpr объекты в коде пользователя оперируют созданными на этапе компиляции масками регистров для работы с периферийными блоками.
	
	Этим достигается высокая производительность. Поскольку программе не нужно <<собирать>> маски регистров в реальном времени, как это сделано в HAL или SPL. Достаточно только применить маску.
	\item Тот факт, что для инициализации глобальных объектов используются глобальные const constexpr структуры вовсе не означает, что данные структуры войдут в состав прошивки контроллера.
	
	Яркий тому пример, объект класса \textit{global\_\-port} (который будет рассмотрен в разделе~\ref{gp:0}). Он принимает в себя массив const constexpr \textit{pin\_config\_t} структур, после чего private constexpr методы объекта класса \textit{global\_\-port} их (структуры) анализируют и возвращают \textit{private global\_\-port\_\-msk\_\-reg\_\-struct} структуру, которая будет private структурой глобального объекта класса \textit{global\_\-port}.
	
	Структуры \textit{pin\_\-config\_\-t}, использовавшиеся для инициализации private \textit{global\_\-port\_\-msk\_\-reg\_\-struct}, во flash загружены не будут, потому что в ходе работы программы обращений к ним не будет.
	\label{kor:0}
	\item Для работы с аппаратной частью контроллера используются объявленные в коде пользователя глобальные const constexpr объекты. В качестве параметра(-ов) конструктора передаётся указатель на const constexpr глобальную(-ые) структуру(-ы) (может передаваться указатель как на одну структуру, так и на массива структур). Важно отметить следующее:
	\begin{itemize}
		\item В случае, если после анализа структур(-ы) инициализации они(-на) больше не требуется - компоновщик не включит эти(-у) структуры(-у) в состав выходного файла программы (о чем было сказано в пункте~\ref{kor:0}). Однако в случае, если используемая структура инициализации, возможно, будет использована во время выполнения программы, как, например, в классе \textit{pin}, описанного в разделе~\ref{pin:0}, то она обязательно пойдёт в состав выходной программы.
		\item Так как конструкторы классов используемых в коде пользователя объектов объявлены внутри класса как constexpr, то создание этих объектов, по сути, заключается в простом копировании в оперативную память их изменяемых данных. Никаких действий в реальном времени (за исключением копирования в оперативную память изменяемых в процессе работы данных объекта) не производится.
		
		Объекты, классы которых требуют вызова функции инициализации  объекта (конструктора) перед вызовом main в реальном времени, \textbf{не поддерживаются намеренно}.
		\item Из того, что все объекты объявлены как const constexpr следует, что у каждого глобального объекта, работающего в реальном времени, имеется метод начальной инициализации (и/или переинициализации), вызов которого необходимо произвести из кода пользователя. 
		
		Это очень оправданно, когда требуется инициализировать объекты в определённом порядке в ходе выполнения программы, чего сложно достигнуть, когда объекты вызываются автоматически перед вызовом функции \textit{main}. Именно это является причиной отказа от поддержки конструкторов классов, вызов функций инициализации (конструкторов) которых, без применения дополнительных директив, производится в случайном порядке (нельзя гарантировать, инициализация какого объекта будет произведена раньше).
		
		\item В случае, если const constexpr объект был объявлен глобально в коде пользователя, но обращений к нему не было на протяжении всей программы, он не будет добавлен в итоговый файл программы. Ситуация здесь аналогична ситуации с глобальными const constexpr структурами.
	\end{itemize}
\end{enumerate}


\label{gp:0}				% global_port начало раздела.
\label{pin:0}				% pin класс в начале описания.
