\section{Модуль работы с портами ввода-вывода общего назначения (PORT)}\label{port:0}
Данный модуль состоит из следующих классов:
\begin{itemize}
	\item \textbf{pin} (раздел~\ref{class:pin});
	\item \textbf{global\_port} (раздел~\ref{class:global:port});
\end{itemize}

\subsection{Класс pin}\label{class:pin}
\subsubsection{Общие сведения о классе}
Объект данного класса позволяет работать с конкретным выводом порта и предоставляет следующие методы:
\begin{itemize}
	\item \textbf{constexpr pin ( const pin\_config\_t* const pin\_cfg\_array,  uint8\_t size = 1 );}\\Конструктор класса принимает указатель на структуру/массив структур конфигурации режима работы  вывода.\\\textbf{Замечание 1: } в случае, если метод \textbf{reinit} объекта будет использован в коде пользователя или драйверах высокого уровня, структура/массив структур, на которую(-ый) ссылается указатель, полностью попадает во flash. В связи с тем, что метод \textbf{reinit} будет использовать структуру/элемент массива.
	\\\textbf{Замечание 2: }важно понимать, что все структуры конфигурации вывода (в случае, если передаётся указатель на массив структур) должны относится к одному физическому выводу (изменяя лишь режим его работы).
	\\\textbf{Замечание 3: }В случае, если в качестве параметра выступая единичная структура, а не массив структур, второй параметр указывать н нужно.
	\item \textbf{void	set		( void ) const;}\\Устанавливает 1 на выводе, если вывод сконфигурирован как выход.
	\item \textbf{void	reset	( void ) const;}\\Устанавливает 0 на выводе, если вывод сконфигурирован как выход.
	\item \textbf{void	set		( uint8\_t state ) const;}\\Устанавливает заданное состояние на выводе (1 или 0), если вывод сконфигурирован как выход.
	\item \textbf{void	set		( bool state ) const;}\\Устанавливает заданное состояние на выводе (true == 1 или false == 0), если вывод сконфигурирован как выход.
	\item \textbf{void	set		( int state ) const;}\\Устанавливает заданное состояние на выводе (1 или 0), если вывод сконфигурирован как выход.
	\item \textbf{void	invert	( void ) const;}\\Логическое \textbf{НЕ} состояния на выходе вывода, если вывод сконфигурирован как выход.
	\item \textbf{int		read	( void ) const;}\\Возвращает логическое состояние на входе вывода, если вывод сконфигурирован как вход.
	\item \textbf{EC\_ANSWER\_PIN\_REINIT	reinit	(uint8\_t number\_config) const;}\\Переинициализирует вывод в ходе выполнения программы в выбранную конфигурацию.\\\textbf{Замечание: }первичная инициализация вывода по нулевой (0) структуре производится объектом \textit{global\_port}. Что избавляет от надобности начальной инициализации вывода. Данный метод следует вызывать, если требуется изменить конфигурацию вывода на иную в ходе выполнения программы (при условии, что он не был заблокирован объектом \textit{global\_port}).\\\textbf{Возвращаемое значение:}
	\begin{itemize}
		\item \textbf{OK} - вывод был успешно переинициализирован.
		\item \textbf{LOCKED} - конфигурация вывода была заблокирована.
		\item \textbf{CFG\_NUMBER\_ERROR} - вы попытались инициализировать вывод несуществующей конфигурацией.
	\end{itemize}
\end{itemize}