\chapter{Объявление структур}\label{struct}
\section{Когда стоит оборачивать данные в структуру?}
Данные следует обернуть в структуру, если:
\begin{itemize}
	\item требуется передавать более одного параметра в конструктор класса;
	\item требуется вернуть из функции более одного параметра;
\end{itemize}
\textbf{Замечание: }перед тем, как оборачивать данные в структуру проверьте, не имеются ли у вас условий, согласно которым данные должны быть обернуты в упакованную структуру (раздел~\ref{struct:p})
\section{Размещение структур}
Прототипы структур должны быть размещены только в \textbf{.h} файлах. Экземпляры - в \textbf{.cpp}.

\section{Оформление структур}
\begin{itemize}
	\item перед первой объявленной структурой размещается комментарий о начале соответствующей области (области структур), обернутый в многострочный комментарий с явно обозначенными границами символами <<*>> в количестве 70 штук. После комментария должна следовать  пустая строка;
	\item перед каждой структурой размещается ее краткое описание, обернутое в многострочный комментарий. После краткого описания пустая строка не ставится;
	\item заголовок структуры следует оформить следующим образом:
	\begin{enumerate}
		\item ключевое слово struct без отступов в начале строки;
		\item отступ в один (1) пробел;
		\item имя структуры;
		\item пробел;
		\item открывающая тело packed структуры скобка <<\{>>;
	\end{enumerate}
	\item поля структуры следует оформлять следующим образом:
	\begin{enumerate}
		\item каждая строка начинается с отступа в один (1) tab;
		\item тип поля;
		\item требуемое количество отступов, выполненных с помощью tab;
		\item имя поля;
		\item <<;>>;
		\item требуемое количество tab;
		\item <<// >> (// + пробел) + одно строчный комментарий.
	\end{enumerate}
	\item все имена полей структуры должны быть выравнены с помощью tab между собой;
	\item после последнего поля структуры следует скобка закрытия тела структуры (<<\}>>);
	\item после последней структуры вставляется пуста строка.
\end{itemize}\textbf{Пример области структур:}\begin{lstlisting}[language=C++, frame=tlBR, basicstyle=\fontsize{10}{10}\ttfamily]
/**********************************************************************
 * Область структур.
 **********************************************************************/

/*
 * Краткое описание структуры...
 */
struct a {
	uint32_t	b;		// Пояснение к полю b.
	uint32_t	c;		// Пояснение к полю c.
};\end{lstlisting}