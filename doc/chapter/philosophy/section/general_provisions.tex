\section{Идеи, лежащие в основе библиотеки}На этапе проектирования библиотеки были отобраны следующие идеи, которые впоследствии легли в ее основу:
\begin{enumerate}
	\item Все, что можно вычислить на этапе компиляции - не должно вычисляться в реальном времени. 
	\item Между производительностью и расходом памяти выбор должен быть в сторону производительности.
	\item Все, что может быть выполнено с помощью аппаратной периферии - не должно выполняться программно.
	\item Библиотека должна иметь как можно больше средств гибкой настройки на этапе компиляции и по минимуму - в реальном времени (в угоду производительности).
	\item Работа программы должна быть по максимуму предсказуема ещё на этапе компиляции. Отсюда следует, что все режимы работы периферии должны быть заданы статически.
	\item Все используемые блоки периферии (как связанные с аппаратной периферией, так и являющиеся логической надстройкой) должны быть объявлены в коде пользователя в виде глобальных const constexpr объектов.
\end{enumerate}