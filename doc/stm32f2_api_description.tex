\documentclass[a4paper, 12pt]{report}		% Документ является статьей на несколько глав.
\usepackage[left=25mm, top=20mm, right=10mm, bottom=25mm, nohead, nofoot]{geometry}
\usepackage [warn]{mathtext}				% Чтобы можно было использовать русские буквы в формулах, 
											% но в случае использования предупреждать об этом
\usepackage{placeins}
\usepackage [T2A]{fontenc}		            % Выбор внутренней TEX−кодировки.
\usepackage [utf8]{inputenc}		        % Выбор кодовой страницы документа.
\usepackage [english, russian]{babel}		% Выбор языка документа.
\usepackage{amsmath}						% Математика.
\usepackage{svg}
\usepackage{graphicx}						% Картинки.
\usepackage{xcolor}
\usepackage{indentfirst}					% Красная строка в начале абзаца.

%Настраиваем гиппер-ссылки.
\usepackage[pdfpagelayout=OneColumn, 		% pdf отображается как сплошная полоса из A4.
			colorlinks=true,				% Не нужно рисовать рамку вокруг ссылок, 
											% но при этом идет выделение цветом.
			linkcolor=blue					% Используем черный цвет для обозначения 
											% гиппер ссылок в оглавлении.					
]{hyperref}									% Запускаем работу с гиппер ссылками.
\setcounter{chapter}{0}						% Счет идет с 1, а не с 0 в оглавлении.

\begin{document}
	%---------------------------------------------------------------------------
	% Титульный лист.
	%---------------------------------------------------------------------------
	\title {ОПИСАНИЕ БИБЛИОТЕКИ STM32F2\_API}
	\author {Автор: Дерябкин Вадим (Vadimatorik)}
	\date {2017}
	\maketitle
	
	%---------------------------------------------------------------------------
	% Введение.
	%---------------------------------------------------------------------------
	\chapter{ВВЕДЕНИЕ}
	В данном документе приводится исчерпывающее описание:
	\begin{itemize}
		\item философии библиотеки (логики построения и использования)
		\item соглашения о написании библиотеки (допустимые синтаксические приемы языка и общие правила написания кода)
		\item примеров использования библиотеки в реальных задачах
	\end{itemize}

	\tableofcontents
	\clearpage							% Первая глава должна идти начиная со следущей страницы.
	
	\chapter{ФИЛОСОФИЯ БИБЛИОТЕКИ}
\subsection{Общие сведения}
В основу библиотеки легли следующие постулаты:
\begin{enumerate}
	\item Все, что можно вычислить на этапе компиляции - не должно вычисляться в реальном времени. 
	\item Между производительностью и расходом памяти выбор должен быть в сторону производительности.
	\item Все, что может быть выполнено с помощью аппаратной периферии - не должно выполняться программно.
	\item Библиотека должна иметь как можно больше средств гибкой настройки на этапе компиляции и по минимуму - в реальном времени (в угоду производительности).
	\item Работа программы должна быть по максимуму предсказуема еще на этапе компиляции. Отсюда следует, что все режимы работы периферии должны быть заданы статически.
\end{enumerate}

\subsection{Краткий обзор реализации}
\begin{enumerate}
	\item Библиотека написана на C++14. 
	\item Большую часть библиотеки составляют constexpr функции, которые обрабатывают заполненные пользователем структуры инициализации периферии на этапе компиляции и создают маски регистров для всевозможных, указанных в структуре инициализации, режимов. В реальном времени созданные из const constexpr структур инициализации глобальные объекты в коде пользователя оперируют созданными на этапе компиляции масками регистров для настройки и работы с периферийными блоками.
	
	Этим достигается высокая производительность. Поскольку программе не нужно <<собирать>> маски регистров в реальном времени, как это сделано в HAL или SPL. Достаточно только применить маску.
	\item Тот факт, что для инициализации глобальных объектов используются глобальные const constexpr структуры вовсе не означает, что данные структуры войдут в состав прошивки контроллера.
	
	Яркий тому пример, объект класса global\_\-port (который будет рассмотрен в разделе~\ref{gp:0}). Он принимает в себя массив const constexpr pin\_config\_t структур, после чего private constexpr методы объекта класса global\_\-port их (структуры) анализируют и возвращают private global\_\-port\_\-msk\_\-reg\_\-struct структуру, которая будет private структурой глобального объекта класса global\_\-port.
	
	Структуры pin\_\-config\_\-t, использовавшиеся для инициализации private global\_\-port\_\-msk\_\-reg\_\-struct, во flash загружены не будут, потому что в ходе работы программы обращений к ним не будет.
\end{enumerate}

\label{gp:0}
	\chapter{СОГЛАШЕНИЕ О НАПИСАНИИ БИБЛИОТЕКИ}
\subsection{Общие положения}
В данной главе будет изложен некоторого рода стандарт, которого следует придерживаться на протяжении всего времени написания кода библиотеки (в идеале, и пользовательского кода тоже).

Стандарт распространяется на:
\begin{enumerate}
	\item Дерево проекта и именование файлов (подраздел~\ref{dn:0}).
\end{enumerate}

\label{dn:0}
\end{document}